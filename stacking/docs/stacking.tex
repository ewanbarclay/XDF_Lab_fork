%%%%%%%%%%%%%%%%%%%%%%%%%%%%%%%%%%%%%%%%%
% Lachaise Assignment
% LaTeX Template
% Version 1.0 (26/6/2018)
%
% This template originates from:
% http://www.LaTeXTemplates.com
%
% Authors:
% Marion Lachaise & François Févotte
% Vel (vel@LaTeXTemplates.com)
%
% License:
% CC BY-NC-SA 3.0 (http://creativecommons.org/licenses/by-nc-sa/3.0/)
%
%%%%%%%%%%%%%%%%%%%%%%%%%%%%%%%%%%%%%%%%%

%----------------------------------------------------------------------------------------
%	PACKAGES AND OTHER DOCUMENT CONFIGURATIONS
%----------------------------------------------------------------------------------------

\documentclass{article}
\usepackage{hyperref}

\newcommand{\hint}{\textbf{Hint:}}


\input{structure.tex} % Include the file specifying the document structure and custom commands

%----------------------------------------------------------------------------------------
%	ASSIGNMENT INFORMATION
%----------------------------------------------------------------------------------------

\title{Stacking tutorial} % Title of the assignment

\author{Stephen Wilkins\\ \texttt{s.wilkins@sussex.ac.uk}} % Author name and email address

\date{University of Sussex --- \today} % University, school and/or department name(s) and a date

%----------------------------------------------------------------------------------------

\begin{document}

\maketitle % Print the title

%----------------------------------------------------------------------------------------
%	INTRODUCTION
%----------------------------------------------------------------------------------------

\section*{Introduction} % Unnumbered section





\begin{figure}\label{fig:source}
	\centering
	\includegraphics[width=0.7\textwidth]{../code/figs/source.pdf}
	\caption{The input source. This is simply a 2D gaussian.}
\end{figure}

\begin{figure}\label{fig:frames}
	\centering
	\includegraphics[width=0.7\textwidth]{../code/figs/frames.pdf}
	\caption{Science (top) and significance (bottom) maps of 5 frames. Each includes a noisey background (generated with \texttt{np.random.randn}) and the source. The first four have the same $\sigma$ while the $\sigma$ of the final frame is 5 times larger. THe source is clearly visible in the first 4 frames but less clear, and possibly undetected, in the final frame. }
\end{figure}

\begin{figure}\label{fig:naive}
	\centering
	\includegraphics[width=0.7\textwidth]{../code/figs/naive.pdf}
	\caption{A naive stack of all 5 frames. This should be slightly worse than the first four frames.}
\end{figure}

\begin{figure}\label{fig:weighted}
	\centering
	\includegraphics[width=0.7\textwidth]{../code/figs/weighted.pdf}
	\caption{A weighted stack of all 5 frames. The source should be noticeably clearer.}
\end{figure}




\end{document}
